\newpage
\section{Conclusions}\label{Sec:Conclusion}

\begin{itemize}

For long years, many researches and paper have suggested that democratic political system is inevitable for economic growth. However, these days, countries which do not have democratic political regime like China and Singapore, receive attention for large economic development. Therefore, we analyze to research weather there is a relation between democracy and income. We use Freedom House Political Right Index and PolityI\hspace{-.1em}V as proxies of democracy and GDP log per capita as an income proxies. We research by using fixed effect model and two-stage least square model and we expected that there is a positive relation between them. However, contrary to our expectation, there is no significant relationship between them in our econometric model, so we could not find significant relation between income and democracy. 

We analyze only post war era because data sets for Freedom House Political Right Index and PolityI\hspace{-.1em}V do not reliable and have enough data before the World War Second. But it might be causal significant relation if we research more long time series data. Also, in many articles suggest that democracy is necessary for economic growth, but there might exist a reverse causal relationship between income and democracy, for instance, income is crucial for democratic political systems.


\end{itemize}
